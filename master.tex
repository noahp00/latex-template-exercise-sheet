\documentclass[headsepline, usegeometry]{scrreprt}

% !! USE LuaLaTeX FOR COMPILING WITH CURRENT PREAMBLE !!

% LTeX: enabled=false

% Cleaner text spacing
\usepackage[final]{microtype}
% Math-Stuff
\usepackage{mathtools}
\usepackage{keytheorems}
\usepackage{nicematrix}
% Nice code
\usepackage[newfloat]{minted}
% Plots
\usepackage{pgfplots}
% Font
\usepackage{fontspec}
\usepackage[warnings-off={mathtools-colon,mathtools-overbracket}]{unicode-math}
\setmainfont{XCharter}
\setmathfont{XCharter Math}
\setmathfont[range=\mathcal]{STIX Two Math}
% German
\usepackage[ngerman]{babel}
% Spacing
\linespread{1.04}
\usepackage{setspace}
\usepackage[a4paper, top=1.6cm, bottom=2.0cm, left=2.45cm, right=2.45cm]{geometry}
% New type area calculation with font and linespread
%\KOMAoptions{DIV=last}
% Header
\usepackage{scrlayer-scrpage}
% Colors and graphics
\usepackage[dvipsnames]{xcolor}
\usepackage{graphicx}
% Nice tables
\usepackage{longtable, booktabs}
% Lists
\usepackage[inline]{enumitem}
% Quotes
\usepackage[autostyle=true]{csquotes}
% Nice numbers and units
\usepackage{siunitx}
% References
\usepackage[hidelinks,linktocpage=true]{hyperref}
\hypersetup{
  linkcolor  = RoyalBlue,
  citecolor  = Green,
  urlcolor   = RoyalBlue,
  colorlinks = true,
}
\usepackage[all]{hypcap}
\usepackage{zref-clever}
\usepackage{bookmark}

% Macros
\DeclareSIUnit\dollar{\$}
\newcommand{\Img}{\operatorname{Im}}
\newcommand{\df}{\operatorname{df}}
\newcommand{\diag}{\operatorname{diag}}
\newcommand{\rg}{\operatorname{rg}}
\newcommand{\Z}{\mathbb{Z}}
\newcommand{\N}{\mathbb{N}}
\newcommand{\C}{\mathbb{C}}
\newcommand{\R}{\mathbb{R}}
\newcommand{\Q}{\mathbb{Q}}
\newcommand{\prim}{\mathbb{P}}
\newcommand{\pot}{\mathcal{P}}
\newcommand{\Llr}{\Longleftrightarrow}
\newcommand{\Lr}{\Longrightarrow}
\newcommand{\Ll}{\Longleftarrow}
\newcommand{\Lra}{\Leftrightarrow}
\newcommand{\Ra}{\Rightarrow}
\newcommand{\La}{\Leftarrow}
\renewcommand{\vec}[1]{\symbfit{#1}}
\newcommand{\mat}[1]{\symbfit{#1}}
\DeclarePairedDelimiter\abs{\lvert}{\rvert}
\DeclarePairedDelimiter\norm{\lVert}{\rVert}

% Solution
\newkeytheoremstyle{deristyle}
{
headfont=\normalfont\itshape,
notebraces={}{},
headpunct={.},
bodyfont=\normalfont,
headformat=\NAME\NOTE,
spaceabove = 0pt,
qed = \(\diamond\),
}
\newkeytheorem{solution}[
Refname={Lösung, Lösungen},
style=deristyle,
name=Lösung
]

% Proof
\newkeytheoremstyle{proofstyle}
{
headfont=\normalfont\itshape,
notebraces={}{},
headpunct={.},
bodyfont=\normalfont,
headformat=\NAME\NOTE,
spaceabove = 0pt,
qed
}
\renewkeytheorem{proof}[
Refname={Beweis,Beweise},
style=proofstyle,
name=Beweis
]

% Exercise
\definecolor{excolor}{RGB}{200, 205, 207}
\newkeytheoremstyle{repstyle}
{
headfont=\normalfont\scshape\bfseries,
notefont=\normalfont\bfseries,
notebraces={--- }{},  
headpunct={.},
bodyfont=\normalfont,
headformat=\NAME~\NUMBER\NOTE,
break,
tcolorbox-no-titlebar={
    enhanced,
    breakable,
    sharp corners,
    colback=excolor!20,
    boxrule=0pt,
    leftrule=2pt,       
    colframe=excolor,
    left=8pt,
    right=8pt,
  }
}
\newkeytheorem{exercise}[
  Refname={Aufgabe, Aufgaben},
  style=repstyle,
  name=Aufgabe,
]


% Code
\setminted{     
    bgcolor = yellow!15,     
    linenos=true,              
    numbersep=5pt,              
    fontsize=\small,            
    breaklines=true,            
    frame=single,                         
    escapeinside=||, 
    framesep=0pt,
    autogobble=true, 
    style=xcode,             
}

% Output (from code)
\newminted{output}{
    bgcolor = gray!15, 
    linenos=false,                                
    fontsize=\small,            
    breaklines=true,            
    frame=single,                         
    escapeinside=||, 
    framesep=0pt,            
    autogobble=true,
    style=bw,   
}

% Long code
\newenvironment{longlisting}{\captionsetup{type=listing}}{}

% Less vertical space after code
\AfterEndEnvironment{minted}{\vspace{-4pt}}

% Caption no colon
\renewcommand*{\captionformat}{\ }

% New caption for figures (only for german)
\renewcaptionname{ngerman}{\figurename}{Abb.}
\renewcommand*{\figureformat}{\bfseries\figurename~\thefigure\autodot}

% New caption for tables (only for german)
\renewcaptionname{ngerman}{\tablename}{Tab.}
\renewcommand*{\tableformat}{\bfseries\tablename~\thetable\autodot}

% New caption for code
\SetupFloatingEnvironment{listing}{name=Code,autorefname=Code}
%\captionsetup[listing]{skip=-2pt,labelfont=bf}
\zcRefTypeSetup{listing}{
Name-sg = Code,
Name-pl = Codes,
}

% Allow page breaks in equations
\allowdisplaybreaks[3]

% Allow page breaks in boxes
\tcbuselibrary{breakable}
\tcbuselibrary{skins}

% Version from pgfplot
\pgfplotsset{compat=1.18}

% Externalization for pgfplots for better compiling time
\usepgfplotslibrary{external}
\tikzset{external/system call={lualatex \tikzexternalcheckshellescape -halt-on-error -interaction=batchmode -jobname "\image" "\texsource"}} % chktex 18
\tikzexternalize

% Otherwise error with tikz-externalize
\tcbset{shield externalize}


% Header
\ohead{Noah Pferdekamp}
\ihead{Einführung in die Computergrafik}
\chead{Blatt 2}
\cfoot{}

\begin{document}

\begin{exercise}
    Es seien zwei Punkte \(p_{0},p_{1} \in \R^{3}\) mit \(\norm{p_{0}}=\norm{p_{1}}\) gegeben.
    Gesucht sind die Parameter einer Rotation, welchen den ersten auf den zweiten Punkt rotiert.
    \begin{enumerate}[label = (\alph*)]
        \item Gib eine Formel an, um den Rotationswinkel \(\alpha\) zwischen \(p_{0}\) und \(p_{1}\) zu bestimmen.\label{ex:1a}
        \item Gib eine Formel an, um die Rotationsachse \(v\) mit \(\norm{v}=1\) zu bestimmen.\label{ex:1b}
    \end{enumerate}  
\end{exercise}
\begin{solution}[\ref{ex:1a}]
    Wir wissen, dass 
    \begin{equation*}
        \langle p_{0},p_{1} \rangle = \norm{p_{0}}\norm{p_{1}}\cos(\alpha).
    \end{equation*}
    Damit gilt
    \begin{equation*}
        \alpha = \arccos\left(\frac{\langle p_{0},p_{1} \rangle}{\norm{p_{0}}\norm{p_{1}}}\right) = \arccos\left(\frac{\langle p_{0},p_{1} \rangle}{\norm{p_{0}}^{2}}\right).
    \end{equation*}
\end{solution}
\begin{solution}[\ref{ex:1b}]
    Um \(p_{0}\) auf \(p_{1}\) zu rotieren, benötigen wir eine Rotationsachse, die orthogonal zu den beiden Punkten ist.
    Dies ist gegeben durch
    \begin{equation*}
        v = \frac{p_{0} \times p_{1}}{\norm{p_{0} \times p_{1}}}.
    \end{equation*}  
\end{solution}
\begin{exercise}   
    Es sei ein Punkt \(p \in \R^{3}\) in homogenen Koordinaten
    \( 
        {\big[
        \begin{matrix}
            x & y & z & 1
        \end{matrix}
        \big]}^{T}
    \)
    gegeben.
    \begin{enumerate}[label = (\alph*)]
        \item Leite eine Matrix \(M_{1}\) her, welche den Punkt \(p\) zuerst um einen Winkel \(\alpha\) um die Achse 
            \(
            \big[
            \begin{matrix}
                0 & 0 & 1
            \end{matrix}
            \big]
            ^{T}
            \)  
            dreht und ihn dann um einen Vektor
            \(
            \big[
            \begin{matrix}
                t_{1} & t_{2} & t_{3}
            \end{matrix}
            \big]
            ^{T}
            \) 
            verschiebt.\label{ex:2a}
        \item Leite eine Matrix \(M_{2}\) her, welche den Punkt \(p\) zuerst um den Vektor 
            \(
            \big[
            \begin{matrix}
                t_{1} & t_{2} & t_{3}
            \end{matrix}
            \big]
            ^{T}
            \) 
            verschiebt und ihn dann um einen Winkel \(\alpha\) um die Achse
            \(
            \big[
            \begin{matrix}
                0 & 0 & 1
            \end{matrix}
            \big]
            ^{T}
            \) 
            dreht.\label{ex:2b}
        \item Beschreibe, in welcher Weise die Reihenfolge der obigen Operationen die jeweilige Transformationsmatrix beeinflusst.\label{ex:2c}
    \end{enumerate}
\end{exercise}
\begin{solution}[\ref{ex:2a}]\label{sol:2a}
    Wir erhalten \(M_{1}\) durch das Produkt \(M_{\operatorname{trans}}\cdot M_{\operatorname{rot}}\) mit
    \begin{equation*}
        \delimiterfactor=1000
        M_{\text{trans}} = 
        \begin{bNiceArray}{ccc|c}[margin]
            \Block{3-3}{\symbfit{I}_{3}} & & & t_{1} \\
            & & & t_{2} \\
            & & & t_{3} \\
            \hline % chktex 44
            0 & 0 & 0 & 1  
        \end{bNiceArray}, \quad
        M_{\text{rot}}
        \begin{bmatrix}
            \cos\alpha & -\sin\alpha & 0 & 0 \\
            \sin\alpha & \hphantom{-}\cos\alpha & 0 & 0 \\
            0 & 0 & 1 & 0 \\
            0 & 0 & 0 & 1 
        \end{bmatrix}.
    \end{equation*}
    Damit ist
    \begin{equation*}
        M_{1} = 
        \begin{bmatrix}
            \cos\alpha & -\sin\alpha & 0 & t_{1} \\
            \sin\alpha & \hphantom{-}\cos\alpha & 0 & t_{2} \\
            0 & 0 & 1 & t_{3} \\
            0 & 0 & 0 & 1 
        \end{bmatrix}.
    \end{equation*}
\end{solution}
\begin{solution}[\ref{ex:2b}]
    Analog zur vorherigen Lösung erhalten wir 
    \begin{equation*}
        M_{2} =
        \begin{bmatrix}
            \cos\alpha & -\sin\alpha & 0 & t_{1}\cos\alpha - t_{2}\sin\alpha \\
            \sin\alpha & \hphantom{-}\cos\alpha & 0 & t_{1}\sin\alpha + t_{2}\cos\alpha \\
            0 & 0 & 1 & t_{3} \\
            0 & 0 & 0 & 1 
        \end{bmatrix}.
    \end{equation*}
\end{solution}
\begin{solution}[\ref{ex:2c}]
    Bei~\ref{ex:2a} erfolgt die Translation unabhängig von der Rotation, wie auch an der letzten Spalte von \(M_{1}\) zu sehen ist.
    Dies ist bei~\ref{ex:2b} nicht mehr der Fall, da die Translation mitrotiert wird. 
\end{solution}
\end{document}